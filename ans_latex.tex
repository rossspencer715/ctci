\documentclass{article}
\usepackage{hyperref}
\usepackage{amsmath}
\usepackage[boxruled,vlined,linesnumbered]{algorithm2e}

\begin{document}
Answers to Cracking the Coding Interview in \LaTeX
by Ross Spencer


\section{Big O}
Big O Additional Problems:

1.1 O($b$)

1.2 O($b$)

1.3 O($1$)

1.4 O($\frac{a}{b}$)

1.5 O(log$_2$($n$))

1.6 O($\sqrt[2]{n}$)

1.7 O($n$) in the case that each node has 1 child in the same direction (degenerate tree).

1.8 O($n$), as you have no heuristics on where the node is located

1.9 O($n^2$) as each copy is $1+2+3+...+n-1 <= n(n) \in$ O($n^2$)

1.10 O(log$_10(n)$), which is equalivent to O(log$_2(n)$) (up to a constant factor for change of base)

1.11 Checking if is in order takes O(s) in size of string s, otherwise makes successive calls to every possible string with $c^s$ possibilities, so O($s*c^s$)

1.12 Total is O($b$log$b$) for mergesort + $a$log$b$ for binary searching b for each int in a. So, O(($a+b$)log$b$).


\pagebreak

\section{Arrays \& Strings}
Chapter 1: Arrays \& Strings
Interview Questions


1.1 
\begin{algorithm}
arr = zeros(26)\;
 \For{char c in string}{
  	
  	\If{arr[int(c)] == 0}{
   		arr[int(c)] += 1\;}
   	\Else{
   		Return False
  		}
  	Return True
 	}
 \caption{IsUnique}

\end{algorithm}

1.2
\begin{algorithm}
\If{len(string1) != len(string2)}{
	Return False\;
}
arr = zeros(26)\;
 \For{char c in string1}{
  	
   	arr[int(c)] += 1\;
 }

 \For{char c in string2}{
  	
   	arr[int(c)] -= 1\;
 }

 \For{int i = 0; i $<$ 26; ++i}{
 \If {arr[i] != 0}{
 Return False\;
 }}

 Return True\;
 
 \caption{IsPermutation}

\end{algorithm}

\pagebreak
1.3
\begin{algorithm}
arr = \lq\rq

 \For{char c in string1}{
  	
   	\If{c == ’ ’}{
   		arr += ’\%20’
   	}
   	\Else{
   		arr += c
   	}

 }

 Return arr\;
 
 \caption{URLify}

\end{algorithm}



1.4
Thought process: a palindrome has a multiple of 2 of all but at most one character (e.g., ...abcdcba..., so d could appear an odd number of times as long as the rest appear an even number of times). Iterate through string and count all chars, then set a boolean flag variable that results in False when multiple characters appear an odd number of times.
\begin{algorithm}
dict count = {}
 \For{char c in string1}{
   		count(c) += 1
   	}

 boolean oddAlreadyPresent = False
 \For{char c in count}{
   		\If{count(c) mod 2 == 1}{
   			\If{oddAlreadyPresent == False}{
   				oddAlreadyPresent = True
   			}
   			\Else{
   				Return False
   			}
   		}
   	}

 Return True
 
 \caption{Palindrome Permutation}

\end{algorithm}

\pagebreak
1.5
\begin{algorithm}
\If{string1 == string2}{
	Return True
}
\If{abs(len(string1)-len(string2))$>$1}{
	Return False
}

diffsFromMissing = 0

dict count = {}
\For{c in string1}{
	count[c] += 1
}

\For{c in string2}{
	//Catch error if not in count by adding 1 to diffsFromMissing
	count[c] -= 1
}

\If{diffsFromMissing $>$ 1}{Return False}

\If{sum(counts) $>$ 1}{Return False}

Return True 
 
 \caption{One Away}

\end{algorithm}

\end{document}
