\documentclass{article}
\usepackage{hyperref}
\usepackage{amsmath}
\usepackage[boxruled,vlined,linesnumbered]{algorithm2e}

\begin{document}
Answers to Cracking the Coding Interview in \LaTeX
by Ross Spencer


\section{Big O}
Big O Additional Problems:

1.1 O($b$)

1.2 O($b$)

1.3 O($1$)

1.4 O($\frac{a}{b}$)

1.5 O(log$_2$($n$))

1.6 O($\sqrt[2]{n}$)

1.7 O($n$) in the case that each node has 1 child in the same direction (degenerate tree).

1.8 O($n$), as you have no heuristics on where the node is located

1.9 O($n^2$) as each copy is $1+2+3+...+n-1 <= n(n) \in$ O($n^2$)

1.10 O(log$_10(n)$), which is equalivent to O(log$_2(n)$) (up to a constant factor for change of base)

1.11 Checking if is in order takes O(s) in size of string s, otherwise makes successive calls to every possible string with $c^s$ possibilities, so O($s*c^s$)

1.12 Total is O($b$log$b$) for mergesort + $a$log$b$ for binary searching b for each int in a. So, O(($a+b$)log$b$).


\pagebreak

\section{Arrays \& Strings}
Chapter 1: Arrays \& Strings
Interview Questions


1.1 
\begin{algorithm}
arr = zeros(26)\;
 \For{char c in string}{
  	
  	\If{arr[int(c)] == 0}{
   		arr[int(c)] += 1\;}
   	\Else{
   		Return False
  		}
  	Return True
 	}
 \caption{IsUnique}

\end{algorithm}

1.2
\begin{algorithm}
\If{len(string1) != len(string2)}{
	Return False\;
}
arr = zeros(26)\;
 \For{char c in string1}{
  	
   	arr[int(c)] += 1\;
 }

 \For{char c in string2}{
  	
   	arr[int(c)] -= 1\;
 }

 \For{int i = 0; i $<$ 26; ++i}{
 \If {arr[i] != 0}{
 Return False\;
 }

 Return True\;
 }
 \caption{IsPermutation}

\end{algorithm}


\end{document}
